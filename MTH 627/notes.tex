\documentclass{article}

\usepackage{preamble}

\title{MTH 627: Advanced PDEs notes}
\author{Madison Phelps \thanks{Oregon State University, phelpmad@oregonstate.edu}}
\date{\today}

\begin{document}

\maketitle

\thispagestyle{empty}

\break

\tableofcontents

\thispagestyle{empty}

\break

\section{9.21.22}

Hello.

\begin{definition}[Semi-norm] Let $V$ be a vector space and define $p:V\to \R$. Then $p$ is a semi-norm if
	\begin{enumerate}
		\item $p$ has absolute homogeneity; $p(cx) = |c| p(x)$ for all $x\in V$ and $c\in\R$, and
		\item $p$ has the triangle inequality; $p(x+y) \leq p(x) + p(y)$ for all $x,y\in V$.
	\end{enumerate}
\end{definition}



\begin{property}[(semi-norms)] If $p: V\to \R$ is a semi-norm, then
	\begin{enumerate}
		\item $p$ has the reverse triangle inequality; $|p(x)-p(y)|\leq p(x-y)$, for all $x,y\in V$. 
		
			\begin{solution}{
				Let $x,y\in V$ and suppose $p$ is a semi-norm on $V$. Using the triangle inequality we compute,
					\[p(x) = p(x-y+y) \leq p(x-y)+p(y)\]
					\[p(y) = p(y-x+x) \leq p(x-y) + p(x),\]
				meaning that
					\[p(x) - p(y) \leq p(x-y) \text{ and } p(y)-p(x) \leq p(x-y)\]
				which results in 
					\[-p(x-y) \leq p(x)-p(y) \leq p(x-y)\]
				by multiplying the second inequality by $-1$. Therefore,  $|p(x)-p(y)|\leq p(x-y)$, for all $x,y\in V$.
			}\end{solution}	
				
		\item $p$ is non-negative $p(x) \geq 0$ for all $x\in V$
	\end{enumerate}
	
	Hello.
\end{property}

Hello.

\end{document}